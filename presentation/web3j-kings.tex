% Copyright 2004 by Till Tantau <tantau@users.sourceforge.net>.
%
% In principle, this file can be redistributed and/or modified under
% the terms of the GNU Public License, version 2.
%
% However, this file is supposed to be a template to be modified
% for your own needs. For this reason, if you use this file as a
% template and not specifically distribute it as part of a another
% package/program, I grant the extra permission to freely copy and
% modify this file as you see fit and even to delete this copyright
% notice. 

\documentclass{beamer}

% There are many different themes available for Beamer. A comprehensive
% list with examples is given here:
% http://deic.uab.es/~iblanes/beamer_gallery/index_by_theme.html
% You can uncomment the themes below if you would like to use a different
% one:
%\usetheme{AnnArbor}
%\usetheme{Antibes}
%\usetheme{Bergen}
%\usetheme{Berkeley}
%\usetheme{Berlin}
%\usetheme{Boadilla}
%\usetheme{boxes}
%\usetheme{CambridgeUS}
\usetheme{Copenhagen}
%\usetheme{Darmstadt}
%\usetheme{default}
%\usetheme{Frankfurt}
%\usetheme{Goettingen}
%\usetheme{Hannover}
%\usetheme{Ilmenau}
%\usetheme{JuanLesPins}
%\usetheme{Luebeck}
%\usetheme{Madrid}
%\usetheme{Malmoe}
%\usetheme{Marburg}
%\usetheme{Montpellier}
%\usetheme{PaloAlto}
%\usetheme{Pittsburgh}
%\usetheme{Rochester}
%\usetheme{Singapore}
%\usetheme{Szeged}
%\usetheme{Warsaw}

\usepackage{listings}
\usepackage{multicol}

\title{Web3j and Blockchain}

% A subtitle is optional and this may be deleted
% \subtitle{Optional Subtitle}

\author{Sebastian Raba \& Joshua Richardson}
% - Give the names in the same order as the appear in the paper.
% - Use the \inst{?} command only if the authors have different
%   affiliation.

\institute[] % (optional, but mostly needed)
{
  Blockchain Platform Engineers\\
  Web3 Labs
}
% - Use the \inst command only if there are several affiliations.
% - Keep it simple, no one is interested in your street address.

\date{King's College Web3j Workshop, 2019}
% - Either use conference name or its abbreviation.
% - Not really informative to the audience, more for people (including
%   yourself) who are reading the slides online

\subject{Blockchain}
% This is only inserted into the PDF information catalog. Can be left
% out. 

% If you have a file called "university-logo-filename.xxx", where xxx
% is a graphic format that can be processed by latex or pdflatex,
% resp., then you can add a logo as follows:

% \pgfdeclareimage[height=0.5cm]{university-logo}{university-logo-filename}
% \logo{\pgfuseimage{university-logo}}

% Delete this, if you do not want the table of contents to pop up at
% the beginning of each subsection:
\AtBeginSubsection[]
{
  \begin{frame}<beamer>{Outline}
    \tableofcontents[currentsection,currentsubsection]
  \end{frame}
}

% Let's get started
\begin{document}

\begin{frame}
  \titlepage
\end{frame}

\begin{frame}{About Me}
	\begin{itemize}
		\item {
		I am Sebastian Raba. I joined crypto space in 2017.
		}
		\item {
			Bachelors in Computer Science with Management in King’s College London. 
		}
		\item {
		Joined Web3 Labs a year and a half ago.
		}
		\item {
		Started off developing Web3j and Web3js support for Quorum interactions. Currently working on  Epirus’s backend. 
		}
	\end{itemize}
\end{frame}

\begin{frame}{Outline}
  \tableofcontents
  % You might wish to add the option [pausesections]
\end{frame}

% Section and subsections will appear in the presentation overview
% and table of contents.
\section{Introduction}

\subsection{Blockchain}

\begin{frame}{Blockchain}
	\begin{center}
		\includegraphics[width=1\linewidth]{Blockchain}
	\end{center}
\end{frame}

\begin{frame}{Blockchain}
	\begin{center}
		\includegraphics[width=1\linewidth]{Blockchain-Timeline}
	\end{center}
\end{frame}

\begin{frame}{Blockchain}
	\begin{center}
		\includegraphics[width=1\linewidth]{Blockchain-in-Financial-Services-Landscape}
	\end{center}
\end{frame}

\subsection{Ethereum}

\begin{frame}{Ethereum}
	\begin{itemize}
		\item {
			Very big distributed computer.
		}
		\item {
			Turing-complete virtual machine.
		}
		\item {
			Public blockchain (mainnet \& testnet).
		}
	\end{itemize}
\end{frame}

\begin{frame}{Ether}
	\begin{itemize}
		\item {
			Fuel of the blockchain.
		}
		\item {
			Massive market capitalization.
		}
		\item {
			Economic incentive to participate in consensus.
		}
		\item {
			Obtained buy mining/trading.
		}
		\item {
			Associated with an address 0x.... and a wallet file.
		}
	\end{itemize}
\end{frame}

\begin{frame}{Smart Contract}
	\begin{itemize}
		\item {
			Computerized contract.
		}
		\item {
			Code + data that lives on the blockchain at an address.
		}
		\item {
			Transactions call functions =$>$ state transition.
		}
	\end{itemize}
\end{frame}

\begin{frame}{Greeter Contract}
	\lstinputlisting[numbers=left,basicstyle=\fontsize{7}{7}\ttfamily]{greeter/Greeter.sol}
\end{frame}

\begin{frame}{Transactions}
	\begin{itemize}
		\item {
			Transfer Ether.
		}
		\item {
			Deploy a smart contract.
		}
		\item {
			Call function of a smart contract.
		}
	\end{itemize}
\end{frame}

\begin{frame}{Transaction}
	\begin{center}
		\includegraphics[width=1\linewidth]{Transaction}
	\end{center}
\end{frame}

\begin{frame}{Integrating with Ethereum}
	\begin{itemize}
		\item {
			Smart contract application binary interface
			encoders/decoders
		}
		\item {
			256 bit numeric types
		}
		\item {
			Multiple transaction types
		}
		\item {
			Wallet management
		}
		\item {
			...
		}
	\end{itemize}
\end{frame}

\begin{frame}{Transaction}
	\begin{center}
		Questions?
	\end{center}
\end{frame}

\section{Web3j}

\subsection{Basics}

\begin{frame}{Web3j}
	\begin{center}
		\includegraphics[width=1\linewidth]{web3j-integration}
	\end{center}
\end{frame}

\begin{frame}{Web3j Features}
	\begin{itemize}
		\item {
			Complete Ethereum JSON-RPC implementation
		}
		\item {
			Ethereum wallet support
		}
		\item {
			Smart contract wrappers
		}
		\item {
			Command line tools
		}
		\item {
			Android compatible
		}
	\end{itemize}
\end{frame}

\begin{frame}{Web3j v3.x}
	\begin{itemize}
		\item {
			Modular
		}
		\item {
			Sync/async \& RX Observable API
		}
		\item {
			ENS support (new!)
		}
		\item {
			Truffle support (new!)
		}
	\end{itemize}
\end{frame}

\begin{frame}{Modules}
	\begin{multicols}{2}
		\begin{itemize}
			\item {
				utils
			}
			\item {
				rlp
			}
			\item {
				abi
			}
			\item {
				tuples
			}	
			\item {
				core
			}
			\item {
				codegen
			}
			\item {
				console (command-line tools)
			}
			\item {
				geth
			}
			\item {
				parity
			}
			\item {
				infura
			}
		\end{itemize}
	\end{multicols}
\end{frame}

\begin{frame}{Web3j Transaction}
	\begin{center}
		\includegraphics[width=1\linewidth]{web3j-graph}
	\end{center}
\end{frame}

\begin{frame}{Questions}
	\begin{center}
		Questions?
	\end{center}
\end{frame}

\subsection{Workshop}

\begin{frame}{Workshop}
	\begin{itemize}
		\item {
			1. Download Web3j
		}
		\item {
			2. Clone workshop repository
		}
		\item {
			3. Deploy smart contract
		}
		\item {
			4. Register event listener on smart contract
		}
		\item {
			5. Send transaction to a friend's smart contract
		}
	\end{itemize}
\end{frame}



% Placing a * after \section means it will not show in the
% outline or table of contents.
\section*{Summary}

\begin{frame}{Summary}mates
  \begin{itemize}
  \item
    Recap on \alert{Ethereum}.
  \item
    Introduction to \alert{Web3j}.
  \end{itemize}
  
  \begin{itemize}
  \item
    Feedback for workshop?
  \end{itemize}
\end{frame}



% All of the following is optional and typically not needed. 
\appendix

\begin{frame}{Further Reading}
    
  \begin{thebibliography}{1}
  \setbeamertemplate{bibliography item}[online]
  
	  \bibitem{ConorSvensson}
		Web3 Labs
	    \newblock Web3j docs.
	    \newblock https://web3j.readthedocs.io/en/latest/
	    
	  \bibitem{OpenZeppelin}
		OpenZeppelin.
	    \newblock OpenZeppelin Solidity library.
	    \newblock https://openzeppelin.org/api/docs/open-zeppelin.html
	    
  	  \bibitem{Web3j gitter}
	    Web3 Labs
	    \newblock Web3j Gitter.
	    \newblock https://gitter.im/web3j/web3j
  
  \end{thebibliography}
\end{frame}

\end{document}


